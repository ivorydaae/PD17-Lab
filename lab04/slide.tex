\documentclass[t]{beamer}

\usetheme{CambridgeUS}

\title{IM1003: Programming Design, Spring 2017  \linebreak Lab 04}
\author[bigelephant29]{Jhih-Bang Hsieh\linebreak \small{bigelephant29}}
\institute{\textbf{National Taiwan University}}
\date{}

\usefonttheme{serif}
\usepackage{xeCJK} 
\usepackage{fontspec}
\setCJKmainfont{DFPHeiMedium-B5}

\usepackage{graphicx}
\graphicspath{{/}}

\usepackage{listings}
\usepackage{color}

\definecolor{dkgreen}{rgb}{0,0.6,0}
\definecolor{gray}{rgb}{0.5,0.5,0.5}
\definecolor{mauve}{rgb}{0.58,0,0.82}

\lstset{
  language=C++,
  showstringspaces=false,
  columns=flexible,
  basicstyle={\linespread{0.6}\scriptsize\ttfamily},
  numbers=left,
  firstnumber=1,
  numberfirstline=true,
  numberstyle=\tiny\color{gray},
  keywordstyle=\color{blue},
  commentstyle=\color{dkgreen},
  stringstyle=\color{mauve},
  breaklines=true,
  breakatwhitespace=true,
  tabsize=4,
  xleftmargin=4em
}

\makeatletter
\setbeamertemplate{footline}{%
    \leavevmode%
    \hbox{%
        \begin{beamercolorbox}[wd=.8\paperwidth,ht=2.25ex,dp=1ex,center]{title in head/foot}%
            \usebeamerfont{title in head/foot}\insertshorttitle
        \end{beamercolorbox}%
        \begin{beamercolorbox}[wd=.2\paperwidth,ht=2.25ex,dp=1ex,right]{date in head/foot}%
            \usebeamerfont{date in head/foot}\insertshortdate{}\hspace*{2em}
            \insertframenumber{} / \inserttotalframenumber\hspace*{2ex} 
        \end{beamercolorbox}}%
        \vskip0pt%
    }
\makeatother

\begin{document}

% First Page
\begin{frame}
  \maketitle
\end{frame}

% Outline Page
\begin{frame}{Outline}
  \begin{itemize}
    \item HW2 Code Review
    \item Array
    \item Function
    \item Practice
  \end{itemize}
\end{frame}

% Section: HW2 Code Review
\section{HW2 Code Review}
\begin{frame}{HW2-3}
  \lstinputlisting{code/2_3_1.cpp}
\end{frame}

\begin{frame}{HW2-3}
  \lstinputlisting{code/2_3_2.cpp}
\end{frame}

\begin{frame}{HW2-3}
  \lstinputlisting{code/2_3_3.cpp}
\end{frame}

\begin{frame}{HW2-4}
  \lstinputlisting{code/2_4_1.cpp}
\end{frame}

% Section: Array
\section{Array}
\begin{frame}{Array}
  \begin{itemize}
    \item 陣列分成靜態陣列、動態陣列。
    \item 前者在編譯時期決定大小,存在 stack 裡面。
    \item 後者在執行時期決定大小,存在 heap 裡面。
  \end{itemize}
\end{frame}

\begin{frame}{Static \& Dynamic Array}
  \begin{itemize}
    \item 靜態陣列速度很快。
    \item 動態陣列大小彈性,控制記憶體用量較容易。
    \item 非全域靜態陣列受 stack size 限制。
  \end{itemize}
\end{frame}

\begin{frame}{Declare a static array}
  \lstinputlisting{code/static_array.cpp}
  \begin{itemize}
    \item 這樣宣告了一個名字叫做 my\_array 的靜態陣列,長度為 256。
    \item index: my\_array[0] - my\_array[255]
    \item 大小無法改變,宣告時無法用變數指定大小。
  \end{itemize}
\end{frame}

\begin{frame}{Declare a dynamic array}
  \lstinputlisting{code/dynamic_array.cpp}
  \begin{itemize}
    \item 這樣宣告了一個名字叫做 my\_array 的指標。
    \item 該指標指向一段長度為 256 的動態陣列。
    \item index: my\_array[0] - my\_array[255]
    \item 大小無法改變,但是宣告時可以用變數指定大小。
    \item 陣列使用完畢之後,須自行指定釋放記憶體(delete)。
  \end{itemize}
\end{frame}

\begin{frame}{Notice}
  \lstinputlisting{code/static_wrong_1.cpp}
  \begin{itemize}
    \item 兩種寫法都是錯的。
    \item 靜態陣列須在編譯時期決定大小,所以不能讓使用者輸入時決定(執行時期)。
  \end{itemize}
\end{frame}

\begin{frame}{Notice}
  \lstinputlisting{code/static_wrong_2.cpp}
  \begin{itemize}
    \item 非全域的靜態陣列存在 stack 裡面,受到 stack size 限制。
    \item stack size 通常不大,所以開一個很大的靜態陣列在 function 內是很危險的。
    \item 將這段程式碼抓到電腦上跑跑看,應該會出問題才對!
    \item 可以在編譯參數中將 stack 調大避免這個問題。
    \item 用動態陣列才是最好的方法!
  \end{itemize}
\end{frame}

\section{Function}
\begin{frame}{Function}
  
\end{frame}

\section{Practice}
\begin{frame}{*Practice G - 2D Rotation}
  給你一個正方形的二維矩陣,請輸出其順時針旋轉 90 度、180度、270度、360度的結果。\\
  輸入、輸出請見 PDOGS。\\
  1 2 3\\
  4 5 6\\
  7 8 9\\
  \vspace{1em}
  旋轉 90 度:\\
  7 4 1\\
  8 5 2\\
  9 6 3
\end{frame}

\end{document}
