\documentclass[t]{beamer}

\usetheme{CambridgeUS}

\title{IM1003: Programming Design, Spring 2017  \linebreak Lab 01}
\author[bigelephant29]{Jhih-Bang Hsieh\linebreak \small{bigelephant29}}
\institute{\textbf{National Taiwan University}}
\date{}

\usefonttheme{serif}
\usepackage{xeCJK} 
\usepackage{fontspec}
\setCJKmainfont{Adobe Ming Std}

\usepackage{graphicx}
\graphicspath{{/}}

\makeatletter
\setbeamertemplate{footline}{%
    \leavevmode%
    \hbox{%
        \begin{beamercolorbox}[wd=.8\paperwidth,ht=2.25ex,dp=1ex,center]{title in head/foot}%
            \usebeamerfont{title in head/foot}\insertshorttitle
        \end{beamercolorbox}%
        \begin{beamercolorbox}[wd=.2\paperwidth,ht=2.25ex,dp=1ex,right]{date in head/foot}%
            \usebeamerfont{date in head/foot}\insertshortdate{}\hspace*{2em}
            \insertframenumber{} / \inserttotalframenumber\hspace*{2ex} 
        \end{beamercolorbox}}%
        \vskip0pt%
    }
\makeatother

\begin{document}

% First Page
\begin{frame}
  \maketitle
\end{frame}

% Outline Page
\begin{frame}{Outline}
  \begin{itemize}
    \item TA \& Lab
    \item Introduction to PDOGS
    \item IDE
    \item Text Editor
    \item Installing Compiler
    \item Coding Style
    \item Practice
  \end{itemize}
\end{frame}

% Section: TA & Lab
\section{TA \& Lab}
\begin{frame}{TA}
  \begin{itemize}
    \setlength\itemsep{1em}
    \item
    李維哲\ Jeff Lee\\
    r04725023(AT)ntu.edu.tw
    \item
    林敬傑\ Jack Lin\\
    r05725007(AT)ntu.edu.tw
    \item
    楊佩蓉\ Sophie Yang\\
    r05725028(AT)ntu.edu.tw
    \item
    謝志邦\ Jhih-Bang Hsieh\\
    b02705021(AT)ntu.edu.tw
  \end{itemize}
\end{frame}

\begin{frame}{Lab}
  \begin{itemize}
    \item 地點:管理學院一號館\ 三樓大電腦教室
    \item 時間:星期三 18:30 - 20:15
    \item Lab 不計入學期成績,可自由參加。
    \item Lab 的練習題應該會很有趣。
    \item 四位助教每週 Lab 都會在,不另開 Office hour。
    \item Lab 的所有資料以及練習題參考程式碼皆會放在助教的 Github。
    \item 有任何錯誤歡迎指正!
    \item Link: \href{https://github.com/bigelephant29/PD17-Lab}{\underline{https://github.com/bigelephant29/PD17-Lab}}
  \end{itemize}
\end{frame}

% Section: Introduction to PDOGS
\section{Introduction to PDOGS}
\begin{frame}{Introduction to PDOGS}
  \begin{itemize}
    \item Programming Design Online Grading System
    \item 由資管系 B02、B03、B04 同學齊心協力開發。
    \item 可以用 @ntu.edu.tw 信箱申請帳號。
    \item 系統應該是 24 小時活著,發現系統不小心死掉請戳戳四位獸醫。
    \item 作業、考試皆在系統上,系統備有自動評分功能。
    \item 請不要用任何方式攻擊系統(包含物理攻擊),不然這門課你可能要修兩次以上。
  \end{itemize}
  \begin{center}
    \includegraphics[width=5em]{pdogs.png}
  \end{center}
\end{frame}

% Section: IDE
\section{IDE}
\begin{frame}{IDE}
  \begin{itemize}
    \item Integrated Development Environment 整合開發環境
    \item 整合了文字編輯器(Text Editor)、編譯器(Compiler)的環境。
    \item 有時還會包含除錯器(Debugger)等輔助開發工具。
    \item 常見的 IDE:Dev-C++, Code::Blocks, Xcode, Eclipse ...等。
  \end{itemize}
  \begin{center}
    \includegraphics[height=2.5em]{dev.png}
    \hspace{1em}
    \includegraphics[height=2.5em]{cb.png}
    \hspace{1em}
    \includegraphics[height=2.5em]{xcode.png}
    \hspace{1em}
    \includegraphics[height=2.5em]{eclipse.png}
  \end{center}
\end{frame}

\begin{frame}{Dev-C++}
  \vspace{0.5em}
  \hspace{2em}
  \includegraphics[height=2.5em]{dev.png}
  \vspace{0.5em}
  \begin{itemize}
    \item 只有 Windows 有,幫 QQ。(Mac 使用者請用 Xcode)
    \item 對初學者非常友善的一個 IDE。
    \item 有文字編輯器、C++ 編譯器。
    \item 方便取得,用起來輕鬆愜意。
    \item 缺點就是有一些 bug,不確定「一些」有多少。
    \item Link: \href{https://sourceforge.net/projects/orwelldevcpp/}{\underline{https://sourceforge.net/projects/orwelldevcpp/}}
  \end{itemize}
\end{frame}

% Section: Text Editor
\section{Text Editor}
\begin{frame}{Text Editor}
  \begin{itemize}
    \item 只有文字編輯功能,可能包含程式碼高亮功能。
    \item 因為沒有包含編譯器,編譯器要自己另外裝。
    \item 常見的文字編輯器有:Vim, Sublime Text, Atom
  \end{itemize}
  \begin{center}
    \includegraphics[height=2.5em]{vim.png}
    \hspace{1em}
    \includegraphics[height=2.5em]{st.png}
    \hspace{1em}
    \includegraphics[height=2.5em]{atom.png}
  \end{center}
\end{frame}

\begin{frame}{Vim}
  \vspace{0.5em}
  \hspace{2em}
  \includegraphics[height=2.5em]{vim.png}
  \vspace{0.5em}
  \begin{itemize}
    \item 功能很陽春的純文字編輯器。
    \item 擴充功能的自由度很高,可以自己寫功能進去。
    \item 在工作站上很常用。
    \item 需要記一堆快捷鍵。
  \end{itemize} 
\end{frame}

\begin{frame}{Sublime Text}
  \vspace{0.5em}
  \hspace{2em}
  \includegraphics[height=2.5em]{st.png}
  \vspace{0.5em}
  \begin{itemize}
    \item UI 很棒的文字編輯器。
    \item 可以自己裝很多插件。
    \item Link: \href{https://www.sublimetext.com/}{\underline{https://www.sublimetext.com/}}
  \end{itemize}
\end{frame}

% Section: Installing g++
\section{Installing Compiler}
\begin{frame}{Installing Compiler}
  \begin{itemize}
    \item 常用的編譯器有 g++、clang 兩種。
    \item 一般 Windows、Linux 會直接裝 g++。
    \item Mac 使用者裝 clang 比較方便。
  \end{itemize}
\end{frame}

\begin{frame}{For Windows User}
  \begin{itemize}
    \item 安裝 MinGW 非常方便!
    \item Link: \href{https://sourceforge.net/projects/mingw/files/}{\underline{https://sourceforge.net/projects/mingw/files/}}
    \item 中間的 Looking for the latest version? 大力的給它戳下去。
    \item 安裝完成後,用 MinGW Installation Manager 把 mingw32-gcc-g++ 這項安裝起來。
    \item 將 MinGW 的安裝路徑加進環境變數 PATH 裡面。
    \item 如果對你來說太過困難,還是用 Dev-C++ 吧!
  \end{itemize}
\end{frame}

\begin{frame}{For Ubuntu User}
  \begin{itemize}
    \item 打開你的 Terminal。
    \item sudo apt-get install g++
    \item 沒了。
  \end{itemize}
\end{frame}

\begin{frame}{For Mac User}
  \begin{itemize}
    \item 安裝 homebrew。
    \item 利用 homebrew 指令將 clang 裝起來(也可以用 homebrew 裝 g++)。
    \item 更多資訊可以 Google 一下,有點複雜,Lab 可以來問助教。
  \end{itemize}
\end{frame}

% Section: Coding Style
\section{Coding Style}
\begin{frame}{Coding Style}
\end{frame}

% Section: Practice
\section{Practice}
\begin{frame}{Practice}
\end{frame}

\end{document}
