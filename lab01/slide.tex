\documentclass[t]{beamer}

\usetheme{CambridgeUS}

\title{IM1003: Programming Design, Spring 2017  \linebreak Lab 01}
\author[bigelephant29]{Jhih-Bang Hsieh\linebreak \small{bigelephant29}}
\institute{\textbf{National Taiwan University}}
\date{}

\usefonttheme{serif}
\usepackage{xeCJK} 
\usepackage{fontspec}
\setCJKmainfont{DFPHeiMedium-B5}

\usepackage{graphicx}
\graphicspath{{/}}

\usepackage{listings}
\usepackage{color}

\definecolor{dkgreen}{rgb}{0,0.6,0}
\definecolor{gray}{rgb}{0.5,0.5,0.5}
\definecolor{mauve}{rgb}{0.58,0,0.82}

\lstset{
  language=C++,
  showstringspaces=false,
  columns=flexible,
  basicstyle={\tiny\ttfamily},
  numbers=left,
  firstnumber=1,
  numberfirstline=true,
  numberstyle=\tiny\color{gray},
  keywordstyle=\color{blue},
  commentstyle=\color{dkgreen},
  stringstyle=\color{mauve},
  breaklines=true,
  breakatwhitespace=true,
  tabsize=4,
  xleftmargin=4em
}

\makeatletter
\setbeamertemplate{footline}{%
    \leavevmode%
    \hbox{%
        \begin{beamercolorbox}[wd=.8\paperwidth,ht=2.25ex,dp=1ex,center]{title in head/foot}%
            \usebeamerfont{title in head/foot}\insertshorttitle
        \end{beamercolorbox}%
        \begin{beamercolorbox}[wd=.2\paperwidth,ht=2.25ex,dp=1ex,right]{date in head/foot}%
            \usebeamerfont{date in head/foot}\insertshortdate{}\hspace*{2em}
            \insertframenumber{} / \inserttotalframenumber\hspace*{2ex} 
        \end{beamercolorbox}}%
        \vskip0pt%
    }
\makeatother

\begin{document}

% First Page
\begin{frame}
  \maketitle
\end{frame}

% Outline Page
\begin{frame}{Outline}
  \begin{itemize}
    \item TA \& Lab
    \item Introduction to PDOGS
    \item IDE
    \item Text Editor
    \item Installing Compiler
    \item Coding Style
    \item Practice
  \end{itemize}
\end{frame}

% Section: TA & Lab
\section{TA \& Lab}
\subsection{TA}
\begin{frame}{TA}
  \begin{itemize}
    \setlength\itemsep{1em}
    \item
    李維哲\ Jeff Lee\\
    r04725023(AT)ntu.edu.tw
    \item
    林敬傑\ Jack Lin\\
    r05725007(AT)ntu.edu.tw
    \item
    楊佩蓉\ Sophie Yang\\
    r05725028(AT)ntu.edu.tw
    \item
    謝志邦\ Jhih-Bang Hsieh\\
    b02705021(AT)ntu.edu.tw
  \end{itemize}
\end{frame}

\subsection{Lab}
\begin{frame}{Lab}
  \begin{itemize}
    \item 地點:管理學院一號館\ 三樓大電腦教室
    \item 時間:星期三 18:30 - 20:15
    \item Lab 不計入學期成績,可自由參加。
    \item Lab 的練習題應該會很有趣。
    \item 四位助教每週 Lab 都會在,不另開 Office hour。
    \item Lab 的所有資料以及練習題參考程式碼皆會放在助教的 Github。
    \item 有任何錯誤歡迎指正!
    \item Link: \href{https://github.com/bigelephant29/PD17-Lab}{\underline{https://github.com/bigelephant29/PD17-Lab}}
  \end{itemize}
\end{frame}

% Section: Introduction to PDOGS
\section{Introduction to PDOGS}
\begin{frame}{Introduction to PDOGS}
  \begin{itemize}
    \item Programming Design Online Grading System
    \item 由資管系 B02、B03、B04 同學齊心協力開發。
    \item 可以用 @ntu.edu.tw 信箱申請帳號。
    \item 系統應該是 24 小時活著,發現系統不小心死掉請戳戳四位助教。
    \item 作業、考試皆在系統上,系統備有自動評分功能。
    \item 請不要用任何方式攻擊系統(包含物理攻擊),但是歡迎回報漏洞。
  \end{itemize}
  \begin{center}
    \includegraphics[width=5em]{image/pdogs.png}
  \end{center}
\end{frame}

% Section: IDE
\section{IDE}
\begin{frame}{IDE}
  \begin{itemize}
    \item Integrated Development Environment 整合開發環境
    \item 整合了文字編輯器(Text Editor)、編譯器(Compiler)的環境。
    \item 有時還會包含除錯器(Debugger)等輔助開發工具。
    \item 常見的 IDE:Dev-C++, Code::Blocks, Xcode, Eclipse ...等。
  \end{itemize}
  \begin{center}
    \includegraphics[height=2.5em]{image/dev.png}
    \hspace{1em}
    \includegraphics[height=2.5em]{image/cb.png}
    \hspace{1em}
    \includegraphics[height=2.5em]{image/xcode.png}
    \hspace{1em}
    \includegraphics[height=2.5em]{image/eclipse.png}
  \end{center}
\end{frame}

\subsection{Dev-C++}
\begin{frame}{Dev-C++}
  \vspace{0.5em}
  \hspace{2em}
  \includegraphics[height=2.5em]{image/dev.png}
  \vspace{0.5em}
  \begin{itemize}
    \item 只有 Windows 有,幫 QQ。(Mac 使用者請用 Xcode)
    \item 對初學者非常友善的一個 IDE。
    \item 有文字編輯器、C++ 編譯器。
    \item 方便取得,用起來輕鬆愜意。
    \item 缺點就是有一些 bug,不確定「一些」有多少。
    \item Link: \href{https://sourceforge.net/projects/orwelldevcpp/}{\underline{https://sourceforge.net/projects/orwelldevcpp/}}
  \end{itemize}
\end{frame}

\subsection{Xcode}
\begin{frame}{Xcode}
  \vspace{0.5em}
  \hspace{2em}
  \includegraphics[height=2.5em]{image/xcode.png}
  \vspace{0.5em}
  \begin{itemize}
    \item 只有 Mac 有,是 Mac 使用者主要的開發工具。
    \item 檔案很大,可能要下載很久。
    \item 新增 project 時,選擇 OS X $\to$ Application $\to$ Command Line Tool $\to$ C++。
  \end{itemize}
\end{frame}

% Section: Text Editor
\section{Text Editor}
\begin{frame}{Text Editor}
  \begin{itemize}
    \item 只有文字編輯功能,可能包含程式碼高亮功能。
    \item 因為沒有包含編譯器,編譯器要自己另外裝。
    \item 常見的文字編輯器有:Vim, Sublime Text, Atom, Visual Studio Code
  \end{itemize}
  \begin{center}
    \includegraphics[height=2.5em]{image/vim.png}
    \hspace{1em}
    \includegraphics[height=2.5em]{image/st.png}
    \hspace{1em}
    \includegraphics[height=2.5em]{image/atom.png}
    \hspace{1em}
    \includegraphics[height=2.5em]{image/vscode.png}
  \end{center}
\end{frame}

\subsection{Vim}
\begin{frame}{Vim}
  \vspace{0.5em}
  \hspace{2em}
  \includegraphics[height=2.5em]{image/vim.png}
  \vspace{0.5em}
  \begin{itemize}
    \item 功能很陽春的純文字編輯器。
    \item 擴充功能的自由度很高,可以自己寫功能進去。
    \item 在工作站上很常用。
    \item 需要記一堆快捷鍵。
  \end{itemize} 
\end{frame}

\subsection{Sublime Text}
\begin{frame}{Sublime Text}
  \vspace{0.5em}
  \hspace{2em}
  \includegraphics[height=2.5em]{image/st.png}
  \vspace{0.5em}
  \begin{itemize}
    \item UI 很棒的文字編輯器。
    \item 可以自己裝很多插件。
    \item Link: \href{https://www.sublimetext.com/}{\underline{https://www.sublimetext.com/}}
  \end{itemize}
\end{frame}

% Section: Installing g++
\section{Installing Compiler}
\begin{frame}{Installing Compiler}
  \begin{itemize}
    \item 常用的編譯器有 g++、clang 兩種。
    \item 一般 Windows、Linux 會直接裝 g++。
    \item Mac 使用者裝 clang 比較方便。
  \end{itemize}
\end{frame}

\subsection{Windows}
\begin{frame}{For Windows User}
  \begin{itemize}
    \item 安裝 MinGW 非常方便!
    \item Link: \href{https://sourceforge.net/projects/mingw/files/}{\underline{https://sourceforge.net/projects/mingw/files/}}
    \item 中間的 Looking for the latest version? 大力的給它戳下去。
    \item 安裝完成後,用 MinGW Installation Manager 把 mingw32-gcc-g++ 這項安裝起來。
    \item 將 MinGW 的安裝路徑加進環境變數 PATH 裡面。
    \item 如果對你來說太過困難,還是用 Dev-C++ 吧!
  \end{itemize}
\end{frame}

\subsection{Ubuntu}
\begin{frame}{For Ubuntu User}
  \begin{itemize}
    \item 打開你的 Terminal。
    \item sudo apt-get install g++
    \item 沒了。
  \end{itemize}
\end{frame}

\subsection{Mac}
\begin{frame}{For Mac User}
  \begin{itemize}
    \item 安裝 homebrew。
    \item 利用 homebrew 指令將 clang 裝起來(也可以用 homebrew 裝 g++)。
    \item Link: \href{https://brew.sh/}{\underline{https://brew.sh/}}
    \item 更多資訊可以 Google 一下,有點複雜,Lab 可以來問助教。
  \end{itemize}
\end{frame}

% Section: Coding Style
\section{Coding Style}
\begin{frame}{Coding Style}
  \begin{itemize}
    \item 好的 coding style 很重要,因為會有其他人需要看你的程式碼。
    \item 未來的你也有可能看不懂現在寫的程式碼。
    \item Coding Style 沒有固定的標準,但是簡潔、整齊、有條理是必備要素。
  \end{itemize}
\end{frame}

\begin{frame}{Q\_\_\_Q}
  \begin{minipage}[t]{.40\textwidth}
    \lstinputlisting[breaklines]{code/bad1.cpp}
  \end{minipage}
  \begin{minipage}[t]{.40\textwidth}
    \lstinputlisting[breaklines]{code/bad2.cpp}
  \end{minipage}
\end{frame}

\begin{frame}{\textasciicircum\_\_\_\textasciicircum}
  \lstinputlisting[breaklines]{code/good.cpp}
\end{frame}

% Section: Practice
\section{Practice}
\begin{frame}{Practice}
  \begin{itemize}
    \item 在未來的助教課時間,會有有趣的練習題。
    \item 除了上課範圍內的題目以外,助教會準備一些需要思考的題目。
    \item 也有可能會出現一些要求程式執行效率的問題。
    \item 較難的題目(會打上星號*)會放上 PDOGS 讓大家練習。
  \end{itemize}
\end{frame}

\subsection{A}
\begin{frame}{Practice A}
  輸入三個非負整數 $X$、$Y$、$Z$。
  \begin{enumerate}
    \item
    如果 $X + Y = Z$ 則輸出 ``A''
    \item
    如果 $X + Z = Y$ 則輸出 ``B''
    \item
    如果 $Y + Z = X$ 則輸出 ``C''
    \item
    如果以上三點皆不符合,請輸出 ``No''
  \end{enumerate}
  \vspace{1em}
  例如:\\
  \begin{itemize}
    \item 輸入 0 0 0,則程式必須輸出 ABC
    \item 輸入 1 2 3,則程式必須輸出 A
    \item 輸入 3 7 1,則程式必須輸出 No
  \end{itemize}
\end{frame}

\subsection{B}
\begin{frame}{Practice B}
  持續輸入數字,每次輸入兩個數字 $N$、$M$,直到兩個數字一樣為止。\\
  當兩個數字一樣的時候,請輸出 ``Finish at $N$''。\\
  \vspace{1em}
  例如:\\
  持續輸入 1 2、3 4、5 6、7 7\\
  程式必須輸出 Finish at 7
\end{frame}

\subsection{C}
\begin{frame}{*Practice C - Greatest Common Divisor}
  輸入兩個數字 $N$、$M$,請輸出兩個數字的最大公因數。\\
  \vspace{1em}
  $1\le N, M\le 10^{9}$\\
  \vspace{1em}
  例如:
  \begin{itemize}
    \item 輸入 1 2,程式必須輸出 1
    \item 輸入 12 18,程式必須輸出 6
    \item 輸入 957 1131,程式必須輸出 87
  \end{itemize}
\end{frame}

\subsection{D}
\begin{frame}{*Practice D - Work at Photoshop}
  你是 Photoshop 的工程師,有一天你需要實作圖片編輯的「裁切」功能。\\
  在畫面中有一張被擺在 $(x_{1},y_{1})$、$(x_{2},y_{2})$ 的矩形圖片,並且該圖片垂直、平行於 $X$、$Y$ 軸。\\
  現在使用者用裁切功能框出了 $(x_{3},y_{3})$、$(x_{4},y_{4})$ 這個矩形,想要請你求出被裁切部分的矩形座標。\\
  請依序輸出左上角、右下角座標,如果沒有裁切到任何一個部分,或是裁切部分面積為 $0$,請輸出 ``No''。\\
  \vspace{1em}
  輸入限制:\\
  $-10^{9}\le x_{i}, y_{i} \le 10^{9}$
\end{frame}

\begin{frame}{*Practice D - Work at Photoshop}
  \begin{center}
    \begin{minipage}[t]{.20\textwidth}
      範例輸入:\\
      0 3\\
      2 1\\
      1 2\\
      3 0\\
      範例輸出:\\
      1 2\\
      2 1
    \end{minipage}
    \begin{minipage}[t]{.40\textwidth}
      \begin{center}
        \includegraphics[height=10em]{image/pd.png}
      \end{center}
    \end{minipage}
  \end{center}
\end{frame}
\end{document}
